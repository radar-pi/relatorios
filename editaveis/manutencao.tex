
\chapter{Manutenção}

\title{Introdução}

A manutenção tem como principal objetivo prolongar a vida útil do radar e garantir o melhor funcionamento dele. Para garantir isso, a equipe decidiu por dividir a manutenção em três tópicos: Manutenção preventiva, corretiva e de procedimentos de instalação e manutenção.

O sistema fotovoltaico deve passar por manutenção regularmente, de forma a garantir uma operação eficiente e impedira ocorrência de problemas futuros. Para isso, recomenda-se olhar as recomendações dos fabricantes dos equipamentos utilizados no sistema e nas normas pertinentes à segurança e á utilização dos equipamentos envolvidos na instalação fotovoltaica.

Como se trata de um sistema fotovoltaico de pequeno porte, a manutenção é relativamente rápida e simples. Os procedimentos de manutenção corretiva exigem conhecimentos mais profundos acerca dos componentes do sistema e geralmente necessitam de peças de reposição, devem ser realizados por pessoas capacitadas.



Afim de previnir qualquer forma de dano ao radar e também de auxiliar a melhor manutenção, o aplicativo \textit{mobile} RaDop irá colaborar tanto com a manutenção preventiva quanto com a corretiva. O \textit{Web App, Dashboard}, também irá garantir melhor funcionamento desse radar e também irá auxiliar à realizar manuntenções corretivas.

A ideia do aplicativo é evitar visitas desnecessárias ao radar e também auxiliar o mantenedor a entender como está o estado desse radar. Dessa forma a pessoa que for realizar a manutenção chegará ao local mais informada. Já o \textit{Dashboard} irá auxiliar na manutenção remota, essa manutenção tem carater de correção.

A manutenção dos sistemas eletrônicos consistem em verificação e validação dos subsistemas, trocas dos cabos de alimentação, dos cabos de dados e dos componentes. Para realização destas tarefas faz-se necessário um multímetro com teste de continuidade, cabos RJ-45, cabos coaxiais, conectores SMA, fonte DC, cabos USB e cabos de comunicação entre os módulos NRF24 e GSM.

A parte estrutural requer verificação visual de possiveis pontos de dano e corrosão causados pelas intempéries. A estrutura deve ser verificada, junto dos outros subsistemas, a fim de garantir o funcionamento adequado dos outros equipamentos. Não é necessario equipamentos especializados para a verificação da integridade estrutural, apenas a inspeção visual é suficiente para verificar a necessidade de substituição das peças, juntamente de EPIs para garantir a seguridade do trabalhador.

A periodicidade mínima das visitas técnicas deve ocorrer a cada 12 meses no mínimo, segundo a portaria Inmetro n.º 156, de 25 de agosto de 2004, demais visitas técnicas serão agendadas conforme necessidade.

\section{Procedimentos de segurança na manutenção e instalação }

Os procedimentos de instalação e manutenção devem ser realizados por pessoas capacitadas conforme a NR-10 e em curso de primeiros socorros. No caso da instalação do painel fotovoltaico no radar, trata-se de um trabalho em altura, de forma que o conhecimento dos procedimentos da NR-35 também é necessário.

O trabalhador capacitado deve incluir o uso de equipamentos de proteção individual (EPI), bem como o uso de ferramentas isoladoras e dos instrumentos de medição. Ao trabalhar com condutores energizados ou próximos a eles deve identificar quais equipamentos e condutores podem estar energizados e qual o seu nível de tensão.

Ao iniciar os trabalhos em locais com instalações elétricas, especialmente com baterias, deve certificar que não está portando objeto pessoal metálico.

Os módulos fotovoltaicos produzem energia elétrica sempre que alguma luz solar incide sobre eles, logo se o objetivo for desenergizar o sistema fotovoltaico, é necessário cobrir o módulo com um material opaco.

Quando for feito alguma troca de componente ou modificação na instalação, o sistema fotovoltaico deve estar desenergizado, para evitar riscos de choques elétricos e curtos circuitos acidentais. Lembrando que a desenergização do sistema fotovoltaico implica em desconexão do painel fotovoltaico e do banco de baterias, o que permite trabalhar com o restante do sistema totalmente desenergizado. Porém, para trabalhos no banco de baterias, não é possível desenergiza-lo, e, no caso do painel, somente quando há obstrução completa da incidência de luz.

Durante a manutenção, o técnico deve manter-se isolado de partes energizadas do circuito ou de pontos de aterramento. Deve-se usar luvas e calçados isolantes durante a manutenção.

Não deve colocar ferramentas ou outros objetos metálicos sobre as baterias para evitar curto circuito.

Antes do início dos trabalhos, o compartimento das baterias deve ser bem ventilado, pois pode haver acumulo de gás hidrogênio liberado durante o carregamento, criando uma atmosfera inflamável e com risco de explosão. 

Recomenda-se que antes de abrir o compartimento de bateria e de fazer contato com a mesma, deve-se tocar em uma superfície aterrada para descarregar a eletricidade estática que pode haver no corpo.

Quando for necessário o manuseio da bateria, recomenda-se o uso de cintas para a sua elevação e estruturas de suporte apropriadas para o transporte, devido ao peso da bateria. 

%\section{Procedimentos e cuidados na operação}

\section{Manutenção preventiva}

Será elaborado um aplicativo de manutenção do radar, o qual irá receber informações do radar como um todo, incluindo a parte do sistema fotovoltaico.

O aplicativo de manutenção deverá levar em consideração a operação do sistema fotovoltaico, os principais pontos são, nível de carga da bateria podendo tomar como referência a tensão da bateria, a verificação da atuação do componente de potência, o controlador de carga e a verificação da potência elétrica produzida e demandada pelo sistema.

Além da informação transmitida pelo aplicativo de manutenção é recomendado fazer inspeções periódicas, já que desta forma pequenos problemas podem ser identificados presencialmente podendo ser notados, caso os dados não sejam notórios.

Além da eficiência energética produzida pelo modulo fotovoltaico, deve-se verificar as condições físicas de cada modulo, certificando-se de que a superfície frontal está integra e limpa, as células não apresentam sinais de rachadura e descoloração, a estrutura de fixação do painel está fixa, sem pontos de corrosão e devidamente aterrada.

Deve-se observar a presença de algum sombreamento causado pelo crescimento de vegetação próxima ao painel, algo que é comum no interior do Brasil.

Para a limpeza dos painéis recomenda-se utilizar uma flanela limpa e água, sem uso de sabão. Deve tomar cuidado para não se apoiar nos módulos e dá preferência para este trabalho no início da manhã e no final da tarde, de forma a evitar possíveis choques térmicos. Perdas de até 10\% no desempenho pode ocorrer caso os módulos possuam muita poeira.

O ângulo de dimensionamento original pode possuir uma tolerância de no máximo 5 graus, podendo ser verificado com um inclinômetro.

Para avaliar o desempenho do gerador fotovoltaico, recomenda-se medir sua tensão de circuito aberto e sua corrente de curto-circuito.

A detecção de pontos quentes no módulo fotovoltaico pode ser feito com o auxílio de uma câmera termográfica infravermelha. Se for encontrado pontos quentes no módulo, deve-se verificar se há sombreamento ou sujeira e eliminá-los. Se não for essa a causa, é possível que se trate de células defeituosas, como células em polarização inversa, ou falha no diodo de desvio ou na solda dos condutores. Podem ser detectados também módulos instalados incorretamente quando estes apresentam em toda a sua superfície temperaturas superiores a outros módulos no mesmo arranjo.

O banco de bateria precisa de uma atenção especial, pois é o componente de menor vida útil e de maiores necessidades de manutenção no sistema. 

A manutenção das baterias inclui a limpeza, aperto de conectores, verificação das condições e do desempenho.

É recomendado manter um registro histórico contendo os dados de tensão de cada elemento da bateria, tensão total do banco e anomalias verificadas.

Deve-se observar cuidadosamente a carcaça de cada bateria em relação a rachaduras, trincas e deformações, que são condições que requerem substituição da bateria. 

Deve-se observar a existência de eletrólito derramado na superfície da bateria ou no chão. Podendo ser sinal de sobrecarga e indicar problema no controlador de carga. Para a limpeza recomenda-se escova macia com água e sabão neutro.

Deve-se verificar os valores dos pontos de regulagem de tensão do controlador de carga com relação às especificações das baterias, temperatura de operação e exigencias do sistema. Deve-se  também observar a ocorrencia de ruídos anormais no controlador de carga. Observar ainda no painel do controlador se há alguma indicação de alarme ativada, informando alguma condição imprópria para o equipamento. Deve-se garantir que o controlador de carga esteja instalado em ambiente fechado, limpo e ventilado.

Afim de garantir um bom estado e funcionamento do radar, o aplicativo RaDop exercerá funções para sua prevenção. O aplicativo não irá substitir a visita técnica, ele serve para auxiliar o mantenedor e  para evitar visitas desnecessárias.

As funções do aplicativo que ajudaram a exercer essa prevenção são:
\begin{itemize}
\item O aplicativo irá enviar notificações para os usuários informando quando a carga da bateria chegar a 20%;
\item O aplicativo irá enviar notificações em caso de falha de comunicação do servidor com o radar;
\item O aplicativo irá mostrar relátorios de dados como por exemplo: Temperatura do equipamento e Operacionalidade;
\end{itemize}

A manutenção preventiva dos equipamentos eletrônicos incluem testes de verificação e validação do funcionamentodo de cada componente.
Para o radar o teste de verificação se dá no envio e transmissão de pulso, o responsável pela manutenção aproxima uma placa metálica na frente do radar e verifica se o pulso recebido é o mesmo do pulso enviado. A validação por sua vez consiste no uso de veículo para aferir a velocidade que o radar capta, para comparar se é a mesma a qual o veículo transitou na área que o radar opera.

A parte estrutural deve ser avaliada visualmente para verificar possiveis pontos de dano e corrosão que possam comprometer o funcionamento correto dos equipamentos, parafusos e presilhas devem ser verificados quanto ao seu estado de corrosão e substituidos em caso de presença avançada de corrosão. As hastes, por serem galvanizadas, não necessitam de substituição preventiva, apenas corretiva em caso de dano por choque mecânico (colisão de veiculo, por exemplo). As caixas de componentes devem ter sua vedação verificada, também de forma visual, com intuito de impedir o acesso de água aos equipamentos eletrônicos. Em caso de borracha desgastada e rachada, fornecendo risco de falha de vedação, ela deverá ser substituída por um novo componente. As carcaças das caixas seguem o mesmo princípio das hastes galvanizadas: devem ser substituídas apenas em caso de dano provocado por choque mecânico. 



\section{Manutenção corretiva}

Para a realização da manutenção corretiva deve-se realizar um orçamento relativo aos custos de reparação do sistema. Após a manutenção corretiva, devem ser realizados procedimentos de inspeção antes da colocação do sistema em operação.
A placa \textit{Nuand} tem um modo de contornar esse problema e para fazer isso será preciso rodar uma rotina no cliente que eles disponibilizam. 

Para os sintomas abaixo em um sistema fotovoltaico é necessário uma ação corretiva:
\begin{itemize}
    \item Nenhum ou baixo fluxo de corrente no carregamento do gerador fotovoltaico.
    \item Baixa tensão no gerador.
    \item Baixa carga na bateria.
    \item Tensão na bateria abaixo do ponto de regulagem de retomada de carregamento.
    \item Tensão na bateria acima do ponto de regulagem de retomada de carregamento
    \item Vazamento de eletrólito nas baterias, coloração diferente, com odor e
sobreaquecimento é necessário a troca das baterias
    \item Controlador não carrega as baterias
    \item Operação irregular do controlador de carga e/ou desconexão inadequada de cargas.
    \item Cargas operando incorretamente ou ineficientemente. 
\end{itemize}

%Para a realização da correção da calibração e configuração do radar será desenvolvido uma rotina que irá corrigir houver uma queda da frequência.

O pulso gerado para ser irradiado pela antena é modulado em quadratura e elevado à frequência de 915 MHz por um mixer. Quando esse sinal é refletido na porta \emph{receiver x (RX)} o sinal é demodulado utilizando um mixer e um defasador. Neste processo sinais de interferências e harmônicos podem ser levadas a frequência de 0Hz, ou seja, sinal DC gerando um deslocamento em amplitude do sinal.

O CI LMS6002 do SDR BladeRFx115 tem um modo de contornar esse problema, para fazer isso será preciso rodar uma rotina no cliente  disponibilizado pela equipe de desenvolvimento de \emph{firmware} do fabricante. O  \textit{Dashboard} fará essa comunicação entre o software e a eletrônica. 

O usuário admnistrador do \textit{Dashboard} irá executar através de uma interface gráfica um \textit{script} que fará com que a \textit{Raspberry} execute essa rotina e estabelize a tensão de \emph{offset}.

O único caso que se faria necessário a manutenção corretiva da estrutura seria o de uma colisão mecânica. O dano deverá ser avaliado e deverá ser determinado quais dos componentes estruturais ainda estão em funcionamento adequado. As caixas e suporte da placa solar podem ser reaproveitadas para utilização em uma nova estrutura caso estejam em perfeito funcionamento, ou seja, sem deformação e impedimento da abertura das portas, podendo ser reformadas à condição inicial e reaproveitadas após o devido reparo, caso seja possível. Recomenda-se, contudo, a substituição total das hastes de suporte principais mesmo em caso de possível readequação do formato, uma vez que não é possivel a determinação da integridade estrutural sem a utilização de equipamentos avançados e cuja utilização seria mais cara que a substituição da haste.