\chapter{Trabalhos Futuros} % (fold)
\label{cha:trabalhos-futuros}

\section{Software}

As atividades desenvolvidas entre o primeiro e o segundo ponto de controle, dentro da equipe de software, estenderam as fases de iniciação de projeto, ou seja, elicitação e refinamento de requisitos e dos sistemas. A partir disto a equipe fez todo levantamento necessário para iniciar a construção do subsistema de software para o projeto do radar. Até a data de entrega do segundo ponto de controle a construção nas três frentes foram iniciadas. Futuramente, para os próximos pontos de controle, vamos reforçar nas práticas de \textit{DevOps} e de uso do repositório da organização, construir os demais microsserviços e fazer todas as evoluções necessárias para a entrega das primeiras verões do aplicativo RaDop e do \textit{Dashboard}. Serão feito os testes de integração com os produtos de eletrônica, principalmente testes da captura e identificação de veículos. Testes de comunicação também serão feitos exaustivamente para avaliar se os produtos planejados irão atender as necessidades do radar. Com a construção e os testes feitos, se o produtos forem validados todos serão integrados para a entrega do radar.

\section{Eletrônica}

Ainda existem muitos trabalhos a serem feitos no projeto. Primeiramente deve ser feito os testes do radar, tanto a transmissão quanto a recepção. E com isso será feitos os testes em campo para captura de veículo, após terminado os testes será feito o cálculo da velocidade. Depois faremos os testes de captura de imagem e o pré processamento que será enviado ao servidor. Será também realizado os testes de comunicação com a câmera e o modulo \emph{Wi-fi}. Outro sistema que passará por testes e validações são os de iluminação e de comunicação da \emph{Blade RF} e a \emph{Raspberry}. Após feito essas validações, será integrado os subsistemas de eletrônica e validado.
% chapter Trabalhos Futuros (end)